\documentclass[12pt,a4paper]{article}
%-----------------------PACKAGES-----------------------%
\usepackage[top=1in,bottom=1in,left=0.5in,right=0.5in]{geometry}
\usepackage{graphicx}
\usepackage{array}
\usepackage{xcolor}
\usepackage{adjustbox}
\usepackage{titlesec}
\usepackage{svg}
\usepackage{lettrine}
\usepackage{xcolor}
\usepackage{url}
\usepackage{hyperref}

\begin{document}
\section*{Zipline}
In Chapter 7, many of you faced issues while installing Zipline, on which I can agree because it is kind of outdated and \textbf{Quantopian} is merged with \textbf{Nasdaq}. Just don't get worried and follow along with me and you will be able to install it locally. Just follow these steps.
\begin{enumerate}
    \item Create a new conda environment with the name Zipline or type in terminal\\
    \textcolor{red}{conda create --name Zipline}
    \item Open your jupyter notebook create the file and copy paste these command and run them in different cells:
    \begin{itemize} 
    \item   \textcolor{red}{!wget 'http://prdownloads.sourceforge.net/ta-lib/ta-lib-0.4.0-src.tar.gz'\\
!tar -xzf ta-lib-0.4.0-src.tar.gz\\
\%cd ta-lib/\\
!./configure - -prefix=/usr/\\
!make\\
!sudo make install\\
\%cd /content/\\
!pip install TA-Lib}
\item\color{red}{!pip install zipline-reloaded}
\item\color{red}{!pip3 install nasdaq-data-link}
\end{itemize}
We are done with the installation of the zipline. Now it's time to set up the API in order to fetch the data from Nasdaq because Quantopain became part of it.
\item Go to the \href{https://data.nasdaq.com/}{Nasdaq}, Create your account and get the API key which will be in your profile section.
\begin{itemize}
    \item open your terminal and type:
    \item  \color{red}{export QUANDL\_API\_KEY=YOUR NASDAQ API KEY}
   \item  \color{red}{ zipline ingest –b quandl}
    
\end{itemize}
This will download the desired bundles from NASDAQ.
\end{enumerate}
\end{document}